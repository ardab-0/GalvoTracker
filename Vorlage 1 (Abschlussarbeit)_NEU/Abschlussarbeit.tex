%------------------------------------------------------------
% LaTeX-Vorlage fuer Abschlussarbeiten am Lehrstuhl fuer Hochfrequenztechnik
%------------------------------------------------------------

\documentclass[
12pt,               										% Schriftgroesse
a4paper,            										% Layout fuer DINA4
german,             										% deutsche Sprache, global
twoside,            										% Layout fuer beidseitigen Druck
headinclude,        										% Kopfzeile wird Seiten-Layouts mit beruecksichtigt
headsepline,        										% horizontale Linie unter Kolumnentitel
plainheadsepline,											% horizontale Linie unter Kolumnentitel auch bei Chapter
BCOR20mm,           										% Korrektur fuer die Bindung
DIV18,              										% DIV-Wert fuer die Erstellung des Satzspiegels, siehe scrguide
parskip=half,       										% Absatzabstand statt Absatzeinzug
openany,            										% Kapitel koennen auf geraden und ungeraden Seiten beginnen
bibliography=totoc,version=first, 							% Literaturverz. wird ins Inhaltsverzeichnis eingetragen
numbers=noenddot,   										% Kapitelnummern immer ohne Punkt
captions=tableheading,version=first, 						% korrekte Abstaende bei Tabellenueberschriften
fleqn,             											% fleqn: Gleichungen links (statt mittig)
listof=totoc,version=first									% Abbildungs- und Tabellenverzeichnis ins Inhaltsverzeichnis
]{scrbook}


%--------------- Packages ----------------
\usepackage[ngerman]{babel} 								% Neue deutsche Trennmuster
\usepackage[utf8]{inputenc} 								% direkte Eingabe von Umlauten & Co. 
\usepackage[T1]{fontenc} 									% T1-Schriften
\usepackage{lmodern}
\usepackage[format=hang,justification=raggedright,singlelinecheck=on,labelfont={bf,small},textfont=small,skip=4pt]{caption} % Captions ausrichten 		
															%%%%%%%%%%%%%%%% singlelinecheck=on : Bildunterschriften ein- und zweizeilig unterschiedlich formatiert
\usepackage[centertags]{amsmath} 							% AMS-Mathematik, centertags zentriert Nummer bei split
\usepackage{bm}												% Paket fuer fett und kursiv
\usepackage{upgreek}										% Paket fuer aufrechte griechische Buchstaben
\usepackage{tabularx}										% erweiterte Tabellen
\usepackage{cite} 											% fuer Zitate
\usepackage{graphicx}            							% zum Einbinden von Grafiken
\usepackage{float}               							% u.a. genaue Plazierung von Gleitobjekten mit H
\usepackage{setspace}            							% Zeilenabstand einstellbar
\usepackage{scrlayer-scrpage}        						% Kopf- und Fusszeilen-Layout 
\usepackage{scrhack}										% Damit kriegt man den Fehler mit den Kopf- und Fusszeilen weg
\usepackage{makeidx}										% Paket fuer das Stichwortverzeichnis
\usepackage[usenames,dvipsnames,svgnames,table]{xcolor}		% Paket fuer die Farben
%\usepackage[absolute]{textpos}								% Für Positionierung FAU-Siegel - obsolet %%%
\usepackage{subcaption}
\usepackage{amssymb}
\usepackage{listings}
\usepackage{trfsigns}
%\usepackage[version-1-compatibility,per=slash,decimalsymbol=comma,loctolang={DE:ngerman}]{siunitx} %%%%%%% veraltet; Ersatz: \sisetup{locale = DE, ...}
\usepackage{siunitx} 										%%%%%%%%%
\sisetup{locale = DE, per-mode = fraction} 					% Darstellung von Dezimalzahlen mit Komma, \per für Bruchstrich %%%%%%%%%

\usepackage[abs]{overpic}
\usepackage[colorlinks = false,pdfpagelabels = true,pdfstartview = FitH,bookmarksopen = true,bookmarksnumbered = true,linkcolor = black,plainpages = false,hypertexnames = false,citecolor = black] {hyperref}																			% Paket fuer das verlinkte PDF
\usepackage{tikz}
\usepackage[european,siunitx]{circuitikz}
\usepackage{pgfplots}

%---- Einstellungen fuer matlab2tikz -------
\pgfplotsset{compat=newest}
\pgfplotsset{plot coordinates/math parser=false}
\pgfplotsset{every linear axis/.append style={/pgf/number format/use comma, /pgf/number format/1000 sep={\,},}}
\pgfplotsset{grid style={dashed,black}, every tick/.style={black}}
\newlength\figureheight
\newlength\figurewidth


%--------------- Sonstiges ----------------

\pagestyle{scrheadings}										% Kopf- und Fusszeile...
\renewcommand{\headfont}{\normalfont\sffamily}    			% Kolumnentitel serifenlos
\renewcommand{\pnumfont}{\normalfont\sffamily}    			% Seitennummern serifenlos
\ihead[]{\headmark}              							% Kopfzeile innen
\ohead[\pagemark]{\pagemark}     							% Kopfzeile aussen
\ifoot[]{} 													% Fusszeile innen
\ofoot[]{}    												% Fusszeile aussen	
\setlength{\headheight}{1.5\baselineskip}
\onehalfspacing												% 1,5 Zeilenabstand
%\typearea[current]{current}        						% Neuberechnung des Satzspiegels mit alten Werten nach Aenderung von Zeilenabstand,etc
\renewcommand{\bibname}{Literatur und Quellen} 				% Literaturverzeichnisbezeichnung
\renewcommand{\figurename}{Abb.}   							% Abbildungsbezeichnung
\renewcommand{\listfigurename}{Abbildungsverzeichnis} 		% Abbildungsverzeichnisbezeichnung
\renewcommand{\captionfont}{\small}							% Bildunterschriften klein kursiv

\addto\captionsngerman{
  \renewcommand{\figurename}{Abb.}							% macht statt Abbildung Abb.
  \renewcommand{\tablename}{Tab.}							% macht statt Tabelle Tab.
}




% Ein paar Vereinfachungen fuer Abkuerzungen und so weiter ------------
% schmale Leerzeichen vereinfachen die Lesbarkeit (optische Gliederung einer zusammenhaengenden Zeichengruppe)

\newcommand{\zB}{z.\,B.} 
\newcommand{\ua}{u.\,a.} % Anwendung dann mit \zB\ etc.
\newcommand{\uU}{u.\,U.}
\newcommand{\su}{s.\,u.}
\newcommand{\dH}{d.\,h.} % \dh gibts leider schon, deshalb \dH\
\newcommand{\iA}{i.\,Allg.}


%----------------------- Beginn des Dokuments -----------------------
\begin{document}
\pagenumbering{Roman}										% Roemische Seitennummerierung
{															%%%
\setlength{\voffset}{-2cm}								%%%%%%%%%%%%%%%%%%%%%%%%%%%%%%
\thispagestyle{empty}
\setcounter{page}{-1}
%%%%%%%%%%%%%%%%% nicht verändern %%%%%%%%%% DO NOT TOUCH %%%%%%%%%%%%%%%
\enlargethispage{2cm}	
\begin{minipage}[t]{.5\textwidth}
	\vspace{0pt}
	\raggedright
	\includegraphics*[height = 2.2 cm]{bilder/LHFT_Logo}
\end{minipage}
%
\hfill
%
\begin{minipage}[t]{.5\textwidth}
	\vspace{0pt}
	\raggedleft
	\includegraphics*[height = 2.4 cm]{bilder/FAU_Logo}
\end{minipage}
%
\vfill
%%%%%%%%%%%%%%%%% nicht verändern %%% ENDE %%%%%%%%%%%%%%%%%%%%%%%%%%%%%%


{\centering
%###### BEARBEITUNGSFELDER ############################################ BEARBEITUNGSFELDER #######
\large{Masterarbeit MA 0000} \\
\Large{Titel}
%Vorsicht: latin1-Codierung kennt kein scharfes s
%#######################################################################################################
\par}


%%%%%%%%%%%%%%%%% nicht verändern %%%%%%%%%% DO NOT TOUCH %%%%%%%%%%%%%%%
\vfill
\vspace{.5cm}
%%%%%%%%%%%%%%%%% nicht verändern %%% ENDE %%%%%%%%%%%%%%%%%%%%%%%%%%%%%%


{\raggedright
\begin{tabbing}
XX \= XXXXXXXXXXX \= XXXXXXXXXXXXXXXXXXXXXX \kill
%###### BEARBEITUNGSFELDER ############################################ BEARBEITUNGSFELDER #######
		\> \textbf{Bearbeiter:} 	\> Arda Buglagil 						\\
 		\>							\>									\\
		\> \textbf{Betreuer:}		\> Prof. 	\\
		\>							\> Betreuer1			\\
		\>							\> Betreuer2			\\
		\>							\> 									\\
		\> \textbf{Ausgabedatum:}	\> 01.04.2023	\\
		\> \textbf{Abgabedatum:}	\> 30.09.2023			
%#######################################################################################################
\end{tabbing}
\par}

\newpage
\thispagestyle{empty}

\cleardoublepage
									% Deckblatt
}															%%%
Ich versichere, dass ich die vorliegende Arbeit ohne fremde Hilfe und ohne Benutzung 
anderer als der angegebenen Quellen angefertigt habe und dass die Arbeit in gleicher 
oder ähnlicher Form noch keiner anderen Prüfungsbehörde vorgelegen hat und von dieser 
als Teil einer Prüfungsleistung angenommen wurde. Alle Ausführungen, die wörtlich oder 
sinngemäß übernommen wurden, sind als solche gekennzeichnet.\\
\\
\begin{table}[!htb]
\centering
\begin{tabularx}{\textwidth}{lXl}

       \hspace{6cm} &  & \hspace{6cm} \\
\cline{1-1}\cline{3-3}
Ort, Datum  &  & Unterschrift
\end{tabularx}
\end{table}									% Erklaerung der Eigenarbeit
\cleardoublestandardpage

\chapter*{Kurzzusammenfassung}

--- deutsche Kurzzusammenfassung ---


\chapter*{Abstract}

--- englische Kurzzusammenfassung ---

									% Abstract
\cleardoublestandardpage

\begin{spacing}{1.15}										% evtl. kleinerer Zeilenabstand im IV, AV, TV
\pdfbookmark[1]{Inhaltsverzeichnis}{toc}					% Inhaltsverzeichnis bei den Lesezeichen rein
\tableofcontents 											% Inhaltsverzeichnis erzeugen

\end{spacing}

\addchap{Symbol- und Abkürzungsverzeichnis}
\label{ch:symbole}

\section*{Symbole}
\begin{tabbing}
XXXXXXXX \= XXXXXXXX \= XXXXXXXXXXXXXXXXXXXXXXXXXXXXXXXXXXXXXXXXXXXXXXXXX \kill
$f$							\> $\si{Hz}$			\> Frequenz \\
$U$							\> $\si{V}$				\> elektrische Spannung \\
$Z$							\> $\si{\ohm}$			\> Impedanz \\

$\omega_0$					\> $\si{1/s}$			\> Mitten-Kreisfrequenz \\
\end{tabbing}

\section*{Schreibweisen}
\begin{tabbing}
XXXXXXXX \= XXXXXXXX \= XXXXXXXXXXXXXXXXXXXXXXXXXXXXXXXXXXXXXXXXXXXXXXXXX \kill
$x(t)$						\> 						\> kontinuierliches Signal \\
$x[k]$						\> 						\> diskretes Signal \\
\end{tabbing}

\section*{Abkürzungen}
\begin{tabbing}
XXXXXXXX \= XXXXXXXX \= XXXXXXXXXXXXXXXXXXXXXXXXXXXXXXXXXXXXXXXXXXXXXXXXX \kill
AC							\>						\> Alternating Current \\
ADC							\> 						\> Analog Digital Converter \\
\end{tabbing}
									% Symbol- und Abkuerzungsverzeichnis
\cleardoublestandardpage

\mainmatter													% Hauptteil beginnt

\chapter{Introduction}
\label{ch:einleitung}

--- Introduction ---									% Einleitung (Bezeichnung passend zur Arbeit!)

\chapter{Grundlagen}
\label{ch:grundlagen}

\section{Aufzählungen}
\label{sec:aufzaehlungen}

\LaTeX ~erlaubt viele verschiedene Formatierungen. Allein bei Aufzählungen sind \textit{description} und \textit{itemize} zu nennen:
\begin{description}
	\item[Ein Stichpunkt]
	mit Beschreibung
	\item[Noch ein Stichpunkt]
	mit noch einer Beschreibung
\end{description}

\begin{itemize}
\item \textit{kursiver Text}
\item \textbf{fetter Text}
\item normaler Text
\item \tiny kleiner Text
\end{itemize}


\section{Verlinkungen und Zitate}
\subsection{Verlinkungen}
\label{subsec:verlinkungen}

Dieses Kapitel hat die Nummer \ref{subsec:verlinkungen}. Referenzen können das gesamte Dokument umfassen und zum Beispiel auch auf Bilder wie \ref{fig:normales_bild} verweisen.\newline
Ein Link aus dem Dokument in das Internet ist mit dem Paket hyperref ebenfalls möglich: \newline\url{https://wch.github.io/latexsheet/}\newline
Unter dieser Adresse findet sich ein gutes \LaTeX ~Befehlsblatt!

\subsection{Zitate}
\label{subsec:zitate}
Zitate ergeben ebenfalls Verlinkungen ins Quellenverzeichnis \cite{rfid_handbuch} und \cite[S.10]{tietze_schenk}.
\begin{quote}
Dies ist ein Zitat zum Test. Es ist an der Einrückung erkennbar. Bei langen Zitaten wird die automatische Einrückung der Folgezeilen sichtbar.
\end{quote}


\section{Einbinden von Bildern}
\label{sec:bilder}

\begin{figure}[!htb]\centering
\ctikzset{bipoles/length=1cm}
\begin{circuitikz}[scale=0.75]

%\draw [help lines] (-1,-2) grid (12,5);
%\filldraw [gray] (0,0) circle (2pt);

\draw (0,2) to[short,o-*] (2,2);
\draw (0,0) to[short,o-*] (2,0);

\draw (2,2) to[R, l=$R_\mathrm{P}$ ,*-*] (2,0);

\draw (2,2) to[short] (4,2);
\draw (2,0) to[short] (4,0);

\draw (4,2) to[american inductor, l=$L_\mathrm{P}$ ,*-*] (4,0);

\draw (4,2) to[short] (6,2);
\draw (4,0) to[short] (6,0);

\draw (6,2) to[C, l=$C_\mathrm{P}$] (6,0);


\draw[->] (-1,0) -- (-1,1) -- (0,1);

\draw (-1,0) node [anchor=north]{$\underline{Y}_\mathrm{P}$};

\end{circuitikz}
\caption{Bild mit Tikz erstellt, Bildunterschrift einzeilig und zentriert.}
\label{fig:bild_tikz}
\end{figure}


\begin{figure}[!htb]\centering
\includegraphics*[width = 6cm]{bilder/grundlagen/RX_Basisbandsignal_IQ_Diagramm}
\caption{Gewöhnliches Bild (hier pdf) und da dies eine zweizeilige Bildunterschrift ist, ist sie linksbündig und die zweite Zeile eingerückt.}
\label{fig:normales_bild}
\end{figure}



\section{Gleichungen}
\label{sec:gleichungen}
Gleichungen wie $a=b+c$ können in einem Fließtext als Inline-Formel auftreten oder als abgesetzte Formel:
\begin{align}
	x = \frac{1+2+i}{2} \, .
	\label{eq:gleichung1}
\end{align}
Abgesetzte Formeln müssen in den Text eingefügt werden wie folgender Satz zeigt. Die Eulerformel, die man in der Form
\begin{align}
	\mathrm{e}^{\mathrm{j}\varphi} = \cos(\varphi) + \mathrm{j} \sin(\varphi) 
	\label{eq:gleichung2}
\end{align}
angeben kann, ist in vielen Formelsammlungen zu finden. \\
Bei den Formeln ist auf ISO-31 und DIN 1338 konformes Setzen zu achten. Dokumente hierzu findet man unter \url{http://www.moritz-nadler.de/formelsatz.pdf} und \url{http://www.et.tu-dresden.de/ifa/fileadmin/user_upload/www_files/richtlinien_sa_da/auszug_din_1338.pdf}

\newpage
\section{Tabellen}

Tabellen können einfach mit der tabular-Umgebung aufgebaut werden.\\
Allerdings sind sie floats (sie ordnen sich automatisch an den besten Platz) und so oft irgendwo unterwegs. Diese Tabelle würde direkt über der Überschrift stehen, obwohl sie darunter definiert wurde. Dies kann mit [!ht] unterdrückt werden, was aber oft nicht sinnvoll ist (wegen den Regeln des Textsatzes). [ht] ist die abgeschwächte Version des Befehls und zu bevorzugen.

	\begin{table}[ht]
		\captionabove{Amateurfunkbänder (Auswahl)} %%%%%%%%%%%%%%%%%%%%%%%%%%%%
		\centering
			\begin{tabular}{| l | l | l |} %Ausrichtung festlegen
					\hline Band & Frequenzen & Nutzungsstatus \\ \hline % Überschriften
					\SI{80}{m} 	& \num{3,5} -- \SI{3,8}{MHz} 		& primär\\
					\SI{40}{m} 	& \num{7} -- \SI{7,1}{MHz} 			& primär\\
					\SI{20}{m} 	& \num{14} -- \SI{14,35}{MHz}		& primär \\
					\SI{17}{m} 	& \num{18,068} -- \SI{18,168}{MHz} 	& primär\\
					\SI{15}{m} 	& \num{21} -- \SI{21,45}{MHz} 		& primär\\
					\SI{10}{m} 	& \num{28} -- \SI{29,7}{MHz} 		& primär\\
					\SI{2}{m}	& \num{144} -- \SI{146}{MHz} 		& primär\\
					\SI{70}{cm}	& \num{430} -- \SI{440}{MHz}		& primär\\
					\SI{23}{cm}	& \num{1240} -- \SI{1300}{MHz} 		& sekundär\\
					\SI{13}{cm}	& \num{2320} -- \SI{2450}{MHz} 		& sekundär\\
					\hline
			\end{tabular} 
%		\caption{Amateurfunkbänder (Auswahl)}
		\label{tab:Bandauswahl}
	\end{table}									% Beginn Hauptteil

\chapter{Zusammenfassung}
\label{ch:zusammenfassung}

--- Zusammenfassung ---							% Zusammenfassung (Bezeichnung passend zur Arbeit!)

\appendix													% Anhang

\chapter{Anhang: Überschrift}
\label{ch:anhang}

--- Anhang ---

										% Anhangsinhalt

%\chapter{Anhang: Lebenslauf}
\label{ch:anhang_lebenslauf}

\renewcommand{\arraystretch}{1}
\begin{table}[!htb]
\centering
\begin{tabularx}{\textwidth}{X l}

Name & Max Mustermann \\
\\
Geburtsdatum & 1. April 1900 \\
Geburtsort & Planet Erde \\
\end{tabularx}
\end{table}
\renewcommand{\arraystretch}{1}								%%%

\
\listoffigures   											% Abbildungsverzeichnis
\listoftables   											% Tabellenverzeichnis

\begin{spacing}{1.5}          								% evtl. kleinerer Zeilenabstand im LV

\bibliographystyle{unsrtdin}		         				% Darstellung numerisch nach Vorkommen im Text
\bibliography{literatur}
\end{spacing}


\end{document}
%------------------------- Ende des Dokuments -----------------------

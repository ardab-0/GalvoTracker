\chapter{Project}
\label{ch:project}

\section{Mapping Steering Mirror Coordinates to Target Plane Coordinates}
\label{sec:Mapping Steering Mirror Coordinates to Target Plane Coordinates}

To direct the laser beam MR-E-2 beam steering mirror is used.
Laser beam is produced by a laser which is stationary. Laser
is positioned to hit the steering mirror in the center. After
the beam hits the mirror, it is directed to required position
by adjusting the position of the mirror.


\begin{figure}[!htb]\centering
    \includegraphics*[width = 6cm]{bilder/project/setup.jpg}
    \caption{Laser, depth camera, and beam steering mirror setup}
    \label{fig:setup}
\end{figure}

Using the mirror to steer the laser beam requires a mapping from
3D world coordinate system to mirror coordinate system. The
mapping calculates the required rotation of the mirror around
its axes to direct the laser beam to intended 3D position.
Mirror coordinate system is defined in [1] as in following figure.




\begin{figure}[!htb]\centering
    \includegraphics*[width = 6cm]{bilder/project/mirror_coordinates.png}
    \caption{Internal mirror coordinate system[1]}
    \label{fig:mirror_coordinates}
\end{figure}


The coordinate system is a Cartesian coordinate system with X and Y axes.
Rotation of the mirror around its horizontal and vertical axes are expressed
as x and y values. The range of both axes are [-1, 1] interval. Mirror
has maximum deflection angle of 25\degree .  $\theta$ is the angle between
incoming and reflected beam.  $\theta$ = +50\degree  corresponds to +1
and $\theta$ = -50\degree  corresponds to -1.  X and Y values are required
to be inside a unit circle in order to be a valid point accessible by the mirror.



\begin{figure}[!htb]\centering
    \includegraphics*[width = 6cm]{bilder/project/mirror_rotation.png}
    \caption{Relation between rotation of the mirror ($\theta$) and the
    mirror coordinate (x) in X-axis[1]}
    \label{fig:mirror_rotation}
\end{figure}



\begin{figure}[!htb]\centering
    \includegraphics*[width = 12cm]{bilder/project/coordinate_systems.png}
    \caption{Mirror coordinate system [1]}
    \label{fig:coordinate_systems}
\end{figure}


In the system setup used in experiments following configuration is used:
\begin{itemize}
    \item $P_{1} = M = C$
    \item $O = P_{1} = M = C$
    \item $d = 0$
    \item Incoming ray is coming in y-z plane.
\end{itemize}


The conversion of coordinates on target plane $(x_{t}, y_{t})$ to mirror
coordinates $(x, y)$ is performed in two main steps. Firstly,
the mirror normal ($n_{m}$) is calculated based on position of the
laser, position of the target plane and $(x_{t}, y_{t})$. In the next
step, corresponding mirror coordinates $(x, y)$ is calculated using $n_{m}$.

\subsection{Computation of mirror normal vector $(n_{m})$}
\label{subsec:compute_mirror_normal}

There are 2 different frames of reference I and T. Reference
frame I is centered around $O$. Its z-axis is aligned with $-n_{m}$ direction
and its y-axis is aligned so that incoming laser beam propagates
through yz plane.  Reference frame T is centered around T.
Its z-axis is aligned with $n_{t}$ direction and its y-axis is aligned
so that reflected laser beam propagates through yz plane.
The transformation between frame of references is done with orthogonal
transformation matrices $A_{IT}$ and $A_{TI}$.



\begin{align*}
    n^{T}_{t}       & =
    \begin{bmatrix}
        0 \\
        0 \\
        1
    \end{bmatrix} &
    r^{T}_{OT}      & =
    \begin{bmatrix}
        0 \\
        0 \\
        -D
    \end{bmatrix}      \\
    r^{T}_{TP_{2}}  & =
    \begin{bmatrix}
        x_{t} \\
        y_{t} \\
        0
    \end{bmatrix} &
    n^{I}_{0}       & =
    \begin{bmatrix}
        0 \\
        0 \\
        1
    \end{bmatrix}      \\
    A_{IT}          & =
    \begin{bmatrix}
        1 & 0           & 0            \\
        0 & cos(\alpha) & -sin(\alpha) \\
        0 & sin(\alpha) & cos(\alpha)
    \end{bmatrix}
\end{align*}


Mirror normal vector is calculated with following steps:

\begin{align*}
    n^{I}_{t}      & = A_{IT} \cdot n^{T}_{t}          \\
    n^{I}_{OT}     & = A_{IT} \cdot r^{T}_{OT}         \\
    r^{I}_{TP_{2}} & = A_{IT} \cdot r^{T}_{TP_{2}}     \\
    r^{I}_{OP_{2}} & = r^{T}_{OT} \cdot r^{I}_{TP_{2}} \\
    n^{I}_{1}      & = normalize(r^{I}_{OP_{2}})       \\
    n^{I}_{m}      & = normalize(n_{1} - n_{0})
\end{align*}

\subsection{Computation of mirror coordinates $(x, y)$ }
\label{subsec:compute_mirror_coordinates}


\begin{align*}
    r^{I}_{C}       & =
    \begin{bmatrix}
        0 \\
        0 \\
        d
    \end{bmatrix} &
    N^{I}_{0}       & =
    \begin{bmatrix}
        0 \\
        0 \\
        -1
    \end{bmatrix}      \\
    n^{T}_{t}       & =
    \begin{bmatrix}
        0 \\
        0 \\
        1
    \end{bmatrix} &
    r^{T}_{OT}      & =
    \begin{bmatrix}
        0 \\
        0 \\
        -D
    \end{bmatrix}      \\
    A_{IT}          & =
    \begin{bmatrix}
        1 & 0 & 0 \\
        0 & 1 & 0 \\
        0 & 0 & 1
    \end{bmatrix}
\end{align*}




\begin{align*}
    t_{1}          & = \frac{ (R^{I}_{C} - r^{I}_{OP_{0}}) \cdot n^{I}_{m} + d }{n^{I}_{0} \cdot n^{I}_{m}} \\
    r^{I}_{OP_{1}} & = r^{I}_{OP_{0}} + t_{1} \cdot n^{I}_{0}                                               \\
    n^{I}_{1}      & = n^{I}_{0} - 2 \cdot ( n^{I}_{0} \cdot n^{I}_{m}) \cdot n^{I}_{m}                     \\
    t_{2}          & = \frac{ (r^{T}_{OT} - r^{I}_{OP_{1}}) \cdot n^{T}_{t}}{n^{I}_{1} \cdot n^{T}_{t}}     \\
    r^{I}_{OP_{2}} & = r^{I}_{OP_{1}} + t_{2} \cdot n^{I}_{1}                                               \\
    x              & = \frac{ r^{I}_{OP_{2}}[0] }{D \cdot tan(50 \degree)}                                  \\
    y              & = \frac{ r^{I}_{OP_{2}}[1] }{D \cdot tan(50 \degree)}
\end{align*}




\section{Position Error Measurement System}
\label{sec:Position Error Measurement System}





\begin{figure}[!htb]\centering
    \includegraphics*[width = 10cm]{bilder/project/error_test_setup.jpg}
    \caption{Position error test setup}
    \label{fig:error_test}
\end{figure}

Measurements are done with a 2-dimensional servo motor system with
a PM400 power meter attached.
Measurements are performed by pointing the laser to a specified
target position and then sweeping the power meter in a linear
trajectory. Power recordings are recorded with recording times
and plotted to find the instance with maximum power. The point
with maximum power gives the center position of the laser.
Power measurements are performed by a Python script which
has a loop with a period of around 0.01s. This sampling
period is not constant throughout the measurements and
might deviate from 0.01s.  For this reason, after the
measurement, cubic interpolation is performed to get
a uniformly sampled signal.


\begin{figure}[!htb]\centering
    \includegraphics*[width = 16cm]{bilder/project/error_test_graph_1.png}
    \caption{Power(mW) - Position(mm) graphs for raw signal and cubic interpolation signal}
    \label{fig:error_test_graph_1}
\end{figure}


Interpolation solves the nonuniform sampling problem.
Obtained signal still has noise in it which might alter
the maximum position. To remove the noise a low pass
filter in the form of a moving average filter is applied.
The figure below depicts the signal before and after the
low pass filter. High-frequency noise components are
removed from the signal while keeping the original signal
mostly intact. This operation produced a smoother signal
which is more suitable for peak finding operation.


\begin{figure}[!htb]\centering
    \includegraphics*[width = 16cm]{bilder/project/error_test_graph_2.png}
    \caption{Power(mW) - Position(mm) graphs for raw signal and cubic interpolation signal}
    \label{fig:error_test_graph_2}
\end{figure}


The recorded power signal is plotted with respect to distance
in order to find the position of the laser's center. This method
gives the position of the laser relative to the initial position
of the servo motor. The figure below shows the power levels with
respect to time and position. Position with maximum power is written
in the fourth graph in millimeters.


\begin{figure}[!htb]\centering
    \includegraphics*[width = 16cm]{bilder/project/error_test_graph_3.png}
    \caption{Power(mW) - Position(mm) graphs for raw signal and cubic interpolation signal}
    \label{fig:error_test_graph_3}
\end{figure}


Measurements are performed with different parameters such as the distance
between the mirror and target plane, different sweep locations.


\subsection{Test results }
\label{subsec:test_results}

The coordinate systems that are used in the steering
mirror and the depth camera don't coincide. They are
rotated and translated versions of each other. A point
in camera coordinate system, c, can be transformed into
a point in mirror coordinate system, m, with the following mapping:


\begin{table}[ht]
    \captionabove{Distance between mirror and target plane is set to 410mm. Actual distance between mirror and target plane is 425mm.} %%%%%%%%%%%%%%%%%%%%%%%%%%%%
    \centering
    \begin{tabular}{| c | c | c | c |} %Ausrichtung festlegen
        \hline Mirror Input x & Mirror Input y & Peak Power Position & Relative Distance \\ \hline % Überschriften
        \SI{-3}{mm}           & \SI{-5}{mm}    & \SI{2.322}{mm}      & \SI{-3.096}{mm}   \\
        \SI{-2.5}{mm}         & \SI{-5}{mm}    & \SI{2.853}{mm}      & \SI{-2.565}{mm}   \\
        \SI{-2}{mm}           & \SI{-5}{mm}    & \SI{3.372}{mm}      & \SI{-2.046}{mm}   \\
        \SI{-1.5}{mm}         & \SI{-5}{mm}    & \SI{3.868}{mm}      & \SI{-1.550}{mm}   \\
        \SI{-1}{mm}           & \SI{-5}{mm}    & \SI{4.392}{mm}      & \SI{-1.026}{mm}   \\
        \SI{-0.5}{mm}         & \SI{-5}{mm}    & \SI{4.905}{mm}      & \SI{-0.513}{mm}   \\
        \SI{0}{mm}            & \SI{-5}{mm}    & \SI{5.418}{mm}      & \SI{0}{mm}        \\
        \SI{0.5}{mm}          & \SI{-5}{mm}    & \SI{5.921}{mm}      & \SI{0.503}{mm}    \\
        \SI{1}{mm}            & \SI{-5}{mm}    & \SI{6.446}{mm}      & \SI{1.028}{mm}    \\
        \SI{1.5}{mm}          & \SI{-5}{mm}    & \SI{6.923}{mm}      & \SI{1.505}{mm}    \\
        \SI{2}{mm}            & \SI{-5}{mm}    & \SI{7.427}{mm}      & \SI{2.009}{mm}    \\
        \SI{2.5}{mm}          & \SI{-5}{mm}    & \SI{7.937}{mm}      & \SI{2.519}{mm}    \\
        \SI{3}{mm}            & \SI{-5}{mm}    & \SI{8.452}{mm}      & \SI{3.034}{mm}    \\

        \hline


        \SI{-3}{mm}           & \SI{0}{mm}     & \SI{2.278}{mm}      & \SI{-3.089}{mm}   \\
        \SI{-2.5}{mm}         & \SI{0}{mm}     & \SI{2.792}{mm}      & \SI{-2.575}{mm}   \\
        \SI{-2}{mm}           & \SI{0}{mm}     & \SI{3.311}{mm}      & \SI{-2.056}{mm}   \\
        \SI{-1.5}{mm}         & \SI{0}{mm}     & \SI{3.829}{mm}      & \SI{-1.538}{mm}   \\
        \SI{-1}{mm}           & \SI{0}{mm}     & \SI{4.327}{mm}      & \SI{-1.040}{mm}   \\
        \SI{-0.5}{mm}         & \SI{0}{mm}     & \SI{4.860}{mm}      & \SI{-0.507}{mm}   \\
        \SI{0}{mm}            & \SI{0}{mm}     & \SI{5.367}{mm}      & \SI{0}{mm}        \\
        \SI{0.5}{mm}          & \SI{0}{mm}     & \SI{5.878}{mm}      & \SI{0.511}{mm}    \\
        \SI{1}{mm}            & \SI{0}{mm}     & \SI{6.379}{mm}      & \SI{1.012}{mm}    \\
        \SI{1.5}{mm}          & \SI{0}{mm}     & \SI{6.880}{mm}      & \SI{1.513}{mm}    \\
        \SI{2}{mm}            & \SI{0}{mm}     & \SI{7.392}{mm}      & \SI{2.025}{mm}    \\
        \SI{2.5}{mm}          & \SI{0}{mm}     & \SI{7.880}{mm}      & \SI{2.513}{mm}    \\
        \SI{3}{mm}            & \SI{0}{mm}     & \SI{8.390}{mm}      & \SI{3.023}{mm}    \\

        \hline


        \SI{-3}{mm}           & \SI{5}{mm}     & \SI{2.391}{mm}      & \SI{-3.097}{mm}   \\
        \SI{-2.5}{mm}         & \SI{5}{mm}     & \SI{2.909}{mm}      & \SI{-2.579}{mm}   \\
        \SI{-2}{mm}           & \SI{5}{mm}     & \SI{3.424}{mm}      & \SI{-2.064}{mm}   \\
        \SI{-1.5}{mm}         & \SI{5}{mm}     & \SI{3.949}{mm}      & \SI{-1.539}{mm}   \\
        \SI{-1}{mm}           & \SI{5}{mm}     & \SI{4.468}{mm}      & \SI{-1.020}{mm}   \\
        \SI{-0.5}{mm}         & \SI{5}{mm}     & \SI{4.974}{mm}      & \SI{-0.514}{mm}   \\
        \SI{0}{mm}            & \SI{5}{mm}     & \SI{5.488}{mm}      & \SI{0}{mm}        \\
        \SI{0.5}{mm}          & \SI{5}{mm}     & \SI{6.013}{mm}      & \SI{0.525}{mm}    \\
        \SI{1}{mm}            & \SI{5}{mm}     & \SI{6.534}{mm}      & \SI{1.046}{mm}    \\
        \SI{1.5}{mm}          & \SI{5}{mm}     & \SI{7.042}{mm}      & \SI{1.554}{mm}    \\
        \SI{2}{mm}            & \SI{5}{mm}     & \SI{7.547}{mm}      & \SI{2.059}{mm}    \\
        \SI{2.5}{mm}          & \SI{5}{mm}     & \SI{8.052}{mm}      & \SI{2.564}{mm}    \\
        \SI{3}{mm}            & \SI{5}{mm}     & \SI{8.561}{mm}      & \SI{3.073}{mm}    \\
        \hline
    \end{tabular}
    %		\caption{Amateurfunkbänder (Auswahl)}
    \label{tab:d410}
\end{table}



\begin{table}[ht]
    \captionabove{Distance between mirror and target plane is set to 425mm. Actual distance between mirror and target plane is 425mm.} %%%%%%%%%%%%%%%%%%%%%%%%%%%%
    \centering
    \begin{tabular}{| c | c | c | c |} %Ausrichtung festlegen
        \hline Mirror Input x & Mirror Input y & Peak Power Position & Relative Distance \\ \hline % Überschriften
        \SI{-3}{mm}           & \SI{-5}{mm}    & \SI{6.723}{mm}      & \SI{-2.999}{mm}   \\
        \SI{-2.5}{mm}         & \SI{-5}{mm}    & \SI{7.225}{mm}      & \SI{-2.497}{mm}   \\
        \SI{-2}{mm}           & \SI{-5}{mm}    & \SI{7.727}{mm}      & \SI{-1.995}{mm}   \\
        \SI{-1.5}{mm}         & \SI{-5}{mm}    & \SI{8.218}{mm}      & \SI{-1.504}{mm}   \\
        \SI{-1}{mm}           & \SI{-5}{mm}    & \SI{8.716}{mm}      & \SI{-1.007}{mm}   \\
        \SI{-0.5}{mm}         & \SI{-5}{mm}    & \SI{9.228}{mm}      & \SI{-0.494}{mm}   \\
        \SI{0}{mm}            & \SI{-5}{mm}    & \SI{9.722}{mm}      & \SI{0.0}{mm}      \\
        \SI{0.5}{mm}          & \SI{-5}{mm}    & \SI{10.212}{mm}     & \SI{0.49}{mm}     \\
        \SI{1}{mm}            & \SI{-5}{mm}    & \SI{10.695}{mm}     & \SI{0.973}{mm}    \\
        \SI{1.5}{mm}          & \SI{-5}{mm}    & \SI{11.191}{mm}     & \SI{1.469}{mm}    \\
        \SI{2}{mm}            & \SI{-5}{mm}    & \SI{11.673}{mm}     & \SI{1.951}{mm}    \\
        \SI{2.5}{mm}          & \SI{-5}{mm}    & \SI{12.151}{mm}     & \SI{2.429}{mm}    \\
        \SI{3}{mm}            & \SI{-5}{mm}    & \SI{12.653}{mm}     & \SI{2.931}{mm}    \\

        \hline


        \SI{-3}{mm}           & \SI{0}{mm}     & \SI{6.788}{mm}      & \SI{-2.956}{mm}   \\
        \SI{-2.5}{mm}         & \SI{0}{mm}     & \SI{7.275}{mm}      & \SI{-2.469}{mm}   \\
        \SI{-2}{mm}           & \SI{0}{mm}     & \SI{7.777}{mm}      & \SI{-1.967}{mm}   \\
        \SI{-1.5}{mm}         & \SI{0}{mm}     & \SI{8.289}{mm}      & \SI{-1.455}{mm}   \\
        \SI{-1}{mm}           & \SI{0}{mm}     & \SI{8.758}{mm}      & \SI{-0.986}{mm}   \\
        \SI{-0.5}{mm}         & \SI{0}{mm}     & \SI{9.248}{mm}      & \SI{-0.496}{mm}   \\
        \SI{0}{mm}            & \SI{0}{mm}     & \SI{9.744}{mm}      & \SI{0.0}{mm}      \\
        \SI{0.5}{mm}          & \SI{0}{mm}     & \SI{10.225}{mm}     & \SI{0.481}{mm}    \\
        \SI{1}{mm}            & \SI{0}{mm}     & \SI{10.711}{mm}     & \SI{0.967}{mm}    \\
        \SI{1.5}{mm}          & \SI{0}{mm}     & \SI{11.207}{mm}     & \SI{1.463}{mm}    \\
        \SI{2}{mm}            & \SI{0}{mm}     & \SI{11.666}{mm}     & \SI{1.922}{mm}    \\
        \SI{2.5}{mm}          & \SI{0}{mm}     & \SI{12.156}{mm}     & \SI{2.412}{mm}    \\
        \SI{3}{mm}            & \SI{0}{mm}     & \SI{12.654}{mm}     & \SI{2.91}{mm}     \\

        \hline


        \SI{-3}{mm}           & \SI{5}{mm}     & \SI{6.849}{mm}      & \SI{-2.978}{mm}   \\
        \SI{-2.5}{mm}         & \SI{5}{mm}     & \SI{7.364}{mm}      & \SI{-2.463}{mm}   \\
        \SI{-2}{mm}           & \SI{5}{mm}     & \SI{7.867}{mm}      & \SI{-1.959}{mm}   \\
        \SI{-1.5}{mm}         & \SI{5}{mm}     & \SI{8.359}{mm}      & \SI{-1.468}{mm}   \\
        \SI{-1}{mm}           & \SI{5}{mm}     & \SI{8.843}{mm}      & \SI{-0.984}{mm}   \\
        \SI{-0.5}{mm}         & \SI{5}{mm}     & \SI{9.34}{mm}       & \SI{-0.487}{mm}   \\
        \SI{0}{mm}            & \SI{5}{mm}     & \SI{9.826}{mm}      & \SI{0.0}{mm}      \\
        \SI{0.5}{mm}          & \SI{5}{mm}     & \SI{10.308}{mm}     & \SI{0.481}{mm}    \\
        \SI{1}{mm}            & \SI{5}{mm}     & \SI{10.824}{mm}     & \SI{0.998}{mm}    \\
        \SI{1.5}{mm}          & \SI{5}{mm}     & \SI{11.304}{mm}     & \SI{1.478}{mm}    \\
        \SI{2}{mm}            & \SI{5}{mm}     & \SI{11.793}{mm}     & \SI{1.967}{mm}    \\
        \SI{2.5}{mm}          & \SI{5}{mm}     & \SI{12.279}{mm}     & \SI{2.453}{mm}    \\
        \SI{3}{mm}            & \SI{5}{mm}     & \SI{12.781}{mm}     & \SI{2.954}{mm}    \\
        \hline
    \end{tabular}
    %		\caption{Amateurfunkbänder (Auswahl)}
    \label{tab:d425}
\end{table}



\begin{table}[ht]
    \captionabove{Distance between mirror and target plane is set to 210mm. Actual distance between mirror and target plane is 225mm.} %%%%%%%%%%%%%%%%%%%%%%%%%%%%
    \centering
    \begin{tabular}{| c | c | c | c |} %Ausrichtung festlegen
        \hline Mirror Input x & Mirror Input y & Peak Power Position & Relative Distance \\ \hline % Überschriften
        \SI{-3}{mm}           & \SI{-5}{mm}    & \SI{6.723}{mm}      & \SI{-2.999}{mm}   \\
        \SI{-2.5}{mm}         & \SI{-5}{mm}    & \SI{7.225}{mm}      & \SI{-2.497}{mm}   \\
        \SI{-2}{mm}           & \SI{-5}{mm}    & \SI{7.727}{mm}      & \SI{-1.995}{mm}   \\
        \SI{-1.5}{mm}         & \SI{-5}{mm}    & \SI{8.218}{mm}      & \SI{-1.504}{mm}   \\
        \SI{-1}{mm}           & \SI{-5}{mm}    & \SI{8.716}{mm}      & \SI{-1.007}{mm}   \\
        \SI{-0.5}{mm}         & \SI{-5}{mm}    & \SI{9.228}{mm}      & \SI{-0.494}{mm}   \\
        \SI{0}{mm}            & \SI{-5}{mm}    & \SI{9.722}{mm}      & \SI{0.0}{mm}      \\
        \SI{0.5}{mm}          & \SI{-5}{mm}    & \SI{10.212}{mm}     & \SI{0.49}{mm}     \\
        \SI{1}{mm}            & \SI{-5}{mm}    & \SI{10.695}{mm}     & \SI{0.973}{mm}    \\
        \SI{1.5}{mm}          & \SI{-5}{mm}    & \SI{11.191}{mm}     & \SI{1.469}{mm}    \\
        \SI{2}{mm}            & \SI{-5}{mm}    & \SI{11.673}{mm}     & \SI{1.951}{mm}    \\
        \SI{2.5}{mm}          & \SI{-5}{mm}    & \SI{12.151}{mm}     & \SI{2.429}{mm}    \\
        \SI{3}{mm}            & \SI{-5}{mm}    & \SI{12.653}{mm}     & \SI{2.931}{mm}    \\

        \hline


        \SI{-3}{mm}           & \SI{0}{mm}     & \SI{6.788}{mm}      & \SI{-2.956}{mm}   \\
        \SI{-2.5}{mm}         & \SI{0}{mm}     & \SI{7.275}{mm}      & \SI{-2.469}{mm}   \\
        \SI{-2}{mm}           & \SI{0}{mm}     & \SI{7.777}{mm}      & \SI{-1.967}{mm}   \\
        \SI{-1.5}{mm}         & \SI{0}{mm}     & \SI{8.289}{mm}      & \SI{-1.455}{mm}   \\
        \SI{-1}{mm}           & \SI{0}{mm}     & \SI{8.758}{mm}      & \SI{-0.986}{mm}   \\
        \SI{-0.5}{mm}         & \SI{0}{mm}     & \SI{9.248}{mm}      & \SI{-0.496}{mm}   \\
        \SI{0}{mm}            & \SI{0}{mm}     & \SI{9.744}{mm}      & \SI{0.0}{mm}      \\
        \SI{0.5}{mm}          & \SI{0}{mm}     & \SI{10.225}{mm}     & \SI{0.481}{mm}    \\
        \SI{1}{mm}            & \SI{0}{mm}     & \SI{10.711}{mm}     & \SI{0.967}{mm}    \\
        \SI{1.5}{mm}          & \SI{0}{mm}     & \SI{11.207}{mm}     & \SI{1.463}{mm}    \\
        \SI{2}{mm}            & \SI{0}{mm}     & \SI{11.666}{mm}     & \SI{1.922}{mm}    \\
        \SI{2.5}{mm}          & \SI{0}{mm}     & \SI{12.156}{mm}     & \SI{2.412}{mm}    \\
        \SI{3}{mm}            & \SI{0}{mm}     & \SI{12.654}{mm}     & \SI{2.91}{mm}     \\

        \hline


        \SI{-3}{mm}           & \SI{5}{mm}     & \SI{6.849}{mm}      & \SI{-2.978}{mm}   \\
        \SI{-2.5}{mm}         & \SI{5}{mm}     & \SI{7.364}{mm}      & \SI{-2.463}{mm}   \\
        \SI{-2}{mm}           & \SI{5}{mm}     & \SI{7.867}{mm}      & \SI{-1.959}{mm}   \\
        \SI{-1.5}{mm}         & \SI{5}{mm}     & \SI{8.359}{mm}      & \SI{-1.468}{mm}   \\
        \SI{-1}{mm}           & \SI{5}{mm}     & \SI{8.843}{mm}      & \SI{-0.984}{mm}   \\
        \SI{-0.5}{mm}         & \SI{5}{mm}     & \SI{9.34}{mm}       & \SI{-0.487}{mm}   \\
        \SI{0}{mm}            & \SI{5}{mm}     & \SI{9.826}{mm}      & \SI{0.0}{mm}      \\
        \SI{0.5}{mm}          & \SI{5}{mm}     & \SI{10.308}{mm}     & \SI{0.481}{mm}    \\
        \SI{1}{mm}            & \SI{5}{mm}     & \SI{10.824}{mm}     & \SI{0.998}{mm}    \\
        \SI{1.5}{mm}          & \SI{5}{mm}     & \SI{11.304}{mm}     & \SI{1.478}{mm}    \\
        \SI{2}{mm}            & \SI{5}{mm}     & \SI{11.793}{mm}     & \SI{1.967}{mm}    \\
        \SI{2.5}{mm}          & \SI{5}{mm}     & \SI{12.279}{mm}     & \SI{2.453}{mm}    \\
        \SI{3}{mm}            & \SI{5}{mm}     & \SI{12.781}{mm}     & \SI{2.954}{mm}    \\
        \hline
    \end{tabular}
    %		\caption{Amateurfunkbänder (Auswahl)}
    \label{tab:d210}
\end{table}



\begin{table}[ht]
    \captionabove{Distance between mirror and target plane is set to 225mm. Actual distance between mirror and target plane is 225mm.} %%%%%%%%%%%%%%%%%%%%%%%%%%%%
    \centering
    \begin{tabular}{| c | c | c | c |} %Ausrichtung festlegen
        \hline Mirror Input x & Mirror Input y & Peak Power Position & Relative Distance \\ \hline % Überschriften
        \SI{-3}{mm}           & \SI{-5}{mm}    & \SI{4.985}{mm}      & \SI{-3.044}{mm}   \\
        \SI{-2.5}{mm}         & \SI{-5}{mm}    & \SI{5.503}{mm}      & \SI{-2.526}{mm}   \\
        \SI{-2}{mm}           & \SI{-5}{mm}    & \SI{6.007}{mm}      & \SI{-2.023}{mm}   \\
        \SI{-1.5}{mm}         & \SI{-5}{mm}    & \SI{6.505}{mm}      & \SI{-1.524}{mm}   \\
        \SI{-1}{mm}           & \SI{-5}{mm}    & \SI{7.034}{mm}      & \SI{-0.995}{mm}   \\
        \SI{-0.5}{mm}         & \SI{-5}{mm}    & \SI{7.536}{mm}      & \SI{-0.494}{mm}   \\
        \SI{0}{mm}            & \SI{-5}{mm}    & \SI{8.029}{mm}      & \SI{0.0}{mm}      \\
        \SI{0.5}{mm}          & \SI{-5}{mm}    & \SI{8.539}{mm}      & \SI{0.509}{mm}    \\
        \SI{1}{mm}            & \SI{-5}{mm}    & \SI{9.045}{mm}      & \SI{1.016}{mm}    \\
        \SI{1.5}{mm}          & \SI{-5}{mm}    & \SI{9.546}{mm}      & \SI{1.517}{mm}    \\
        \SI{2}{mm}            & \SI{-5}{mm}    & \SI{10.045}{mm}     & \SI{2.015}{mm}    \\
        \SI{2.5}{mm}          & \SI{-5}{mm}    & \SI{10.529}{mm}     & \SI{2.499}{mm}    \\
        \SI{3}{mm}            & \SI{-5}{mm}    & \SI{11.009}{mm}     & \SI{2.98}{mm}     \\

        \hline


        \SI{-3}{mm}           & \SI{0}{mm}     & \SI{4.989}{mm}      & \SI{-2.992}{mm}   \\
        \SI{-2.5}{mm}         & \SI{0}{mm}     & \SI{5.493}{mm}      & \SI{-2.488}{mm}   \\
        \SI{-2}{mm}           & \SI{0}{mm}     & \SI{5.999}{mm}      & \SI{-1.982}{mm}   \\
        \SI{-1.5}{mm}         & \SI{0}{mm}     & \SI{6.49}{mm}       & \SI{-1.491}{mm}   \\
        \SI{-1}{mm}           & \SI{0}{mm}     & \SI{6.998}{mm}      & \SI{-0.984}{mm}   \\
        \SI{-0.5}{mm}         & \SI{0}{mm}     & \SI{7.459}{mm}      & \SI{-0.523}{mm}   \\
        \SI{0}{mm}            & \SI{0}{mm}     & \SI{7.981}{mm}      & \SI{0.0}{mm}      \\
        \SI{0.5}{mm}          & \SI{0}{mm}     & \SI{8.461}{mm}      & \SI{0.479}{mm}    \\
        \SI{1}{mm}            & \SI{0}{mm}     & \SI{8.952}{mm}      & \SI{0.971}{mm}    \\
        \SI{1.5}{mm}          & \SI{0}{mm}     & \SI{9.485}{mm}      & \SI{1.503}{mm}    \\
        \SI{2}{mm}            & \SI{0}{mm}     & \SI{9.976}{mm}      & \SI{1.994}{mm}    \\
        \SI{2.5}{mm}          & \SI{0}{mm}     & \SI{10.464}{mm}     & \SI{2.483}{mm}    \\
        \SI{3}{mm}            & \SI{0}{mm}     & \SI{10.956}{mm}     & \SI{2.975}{mm}    \\

        \hline


        \SI{-3}{mm}           & \SI{5}{mm}     & \SI{5.156}{mm}      & \SI{-2.982}{mm}   \\
        \SI{-2.5}{mm}         & \SI{5}{mm}     & \SI{5.664}{mm}      & \SI{-2.475}{mm}   \\
        \SI{-2}{mm}           & \SI{5}{mm}     & \SI{6.153}{mm}      & \SI{-1.986}{mm}   \\
        \SI{-1.5}{mm}         & \SI{5}{mm}     & \SI{6.649}{mm}      & \SI{-1.49}{mm}    \\
        \SI{-1}{mm}           & \SI{5}{mm}     & \SI{7.154}{mm}      & \SI{-0.985}{mm}   \\
        \SI{-0.5}{mm}         & \SI{5}{mm}     & \SI{7.64}{mm}       & \SI{-0.498}{mm}   \\
        \SI{0}{mm}            & \SI{5}{mm}     & \SI{8.138}{mm}      & \SI{0.0}{mm}      \\
        \SI{0.5}{mm}          & \SI{5}{mm}     & \SI{8.634}{mm}      & \SI{0.496}{mm}    \\
        \SI{1}{mm}            & \SI{5}{mm}     & \SI{9.126}{mm}      & \SI{0.988}{mm}    \\
        \SI{1.5}{mm}          & \SI{5}{mm}     & \SI{9.635}{mm}      & \SI{1.497}{mm}    \\
        \SI{2}{mm}            & \SI{5}{mm}     & \SI{10.126}{mm}     & \SI{1.988}{mm}    \\
        \SI{2.5}{mm}          & \SI{5}{mm}     & \SI{10.609}{mm}     & \SI{2.47}{mm}     \\
        \SI{3}{mm}            & \SI{5}{mm}     & \SI{11.103}{mm}     & \SI{2.965}{mm}    \\
        \hline
    \end{tabular}
    %		\caption{Amateurfunkbänder (Auswahl)}
    \label{tab:d225}
\end{table}



\section{Mapping Depth Camera Coordinate System to Steering Mirror Coordinate System}
\label{sec:Mapping Depth Camera Coordinate System to Steering Mirror Coordinate System}


The coordinate systems that are used in the steering
mirror and the depth camera don't coincide. They are
rotated and translated versions of each other. A point
in camera coordinate system, c, can be transformed into
a point in mirror coordinate system, m, with the following mapping:

\begin{align*}
    m = R c + t
\end{align*}


\begin{figure}[!htb]\centering
    \includegraphics*[width = 16cm]{bilder/project/mapping1.png}
    \caption{Recording of the same points in steering mirror and depth camera coordinate systems }
    \label{fig:mapping1}
\end{figure}


R is $3x3$ an orthogonal rotation matrix and t is a $3x1$ translation 
vector. Optimal R and t are found by the following algorithm [2] [3] [4]:
R is an orthogonal rotation matrix and t is a translation vector. 
Optimal R and t are found by the following algorithm [2] [3] [4]:


To find the rotation matrix R,

\begin{align*}
    H &= (A-m_{A})(B-m_{B})^{T} \\
    U, S, V^{T} &= SVD(H) \\
    R &= VU^{T}
\end{align*}


To find translation vector t,

\begin{align*}
    t = m_{B} - R m_{A}
\end{align*}



Optimal transformation maps each point in camera coordinate 
system to a point in mirror coordinate system. After the 
mapping mirror points and camera points are almost aligned 
as shown in Figure 8. With this mapping, any point recorded 
by depth camera can be transformed into a point which can be 
used by the mirror controller.


\begin{figure}[!htb]\centering
    \includegraphics*[width = 16cm]{bilder/project/mapping2.png}
    \caption{Camera points after the transformation }
    \label{fig:mapping2}
\end{figure}


\section{Calibration Procedure}
\label{sec:Calibration Procedure}


Calibration of the system is done in order to coordinate depth 
camera and mirror together. Both steering mirror and depth 
camera have their own coordinate systems. The relationship 
between these coordinate systems is unknown which makes them 
unable to perform tasks together. The main objective of the 
calibration is to find the rotation and translation of the 
coordinate systems relative to each other.  For this purpose, 
the optimal R and t finding algorithm discussed in previous 
chapter is used. 
Calibration procedure is responsible for obtaining point pairs 
consisting of a point in depth camera coordinate system and its 
corresponding point in steering mirror coordinate system. After 
the collection of the dataset, the mapping is computed and saved 
for later usage. Point pair collection operation depends on the 
type of laser used in the system. For easier testing a visible 
laser is used. However, in the end system an IR laser is used. 
Using IR laser restricts the usage of camera since it is not 
visible by camera sensor. IR laser requires IR light intensity 
sensors to detect and a different calibration procedure than 
visible laser.  


\subsection{Visible Laser}
\label{sec:Visible Laser}

The laser is pointed at 30 different positions in a plane 
560 mm away from the mirror center. These points are recorded 
as mirror points B. Each laser position is extracted from the 
color images by using Hough Circle Detection algorithm. Once 
2D pixel coordinates are found they are converted to 3D 
coordinates by Kinect Azure SDK's k4a::calibration::convert\_2d\_to\_3d [5] 
function. Extracted 3D depth camera points are named A. 
Points A and B are formed into matrices with shape 3xN. 
Finally, algorithm 1 is used to find R and t matrices


\subsection{IR Laser}
\label{sec:IR Laser}

IR laser requires different calibration method from the 
visible laser. Point patching operation is done via 
laser intensity sensors. The sensor setup used to detect 
positions is in figure below. 3 sensors generate analog 
output signals based on the intensity of light hitting 
them. The analog signal is captured by ADCs on a Raspberry 
Pi Pico board and transmitted to host computer via USB.
--add sensor picture
While the sensor is recording the signal intensity, 
laser scans the area near the sensor position based on an 
initial sensor position estimate in terms of a 3D coordinate. 
Each signal point is plotted with respect to position of the 
laser at the time of recording. As a result, an intensity 
profile is generated. The maximum intensity level of this 
profile is chosen as the sensor position. This process is 
performed by all three sensors, and 3 different images 
are generated. In each image a peak intensity is observed, 
and it is marked as that specific sensor's position.



\begin{figure}[!htb]\centering
    \includegraphics*[width = 6cm]{bilder/project/sensor_im1.png}
    \caption{Rough sensor image.}
    \label{fig:sensor_rough}

    \includegraphics*[width = 6cm]{bilder/project/sensor_im2.png}
    \caption{Fine sensor image.}
    \label{fig:sensor_rough}
\end{figure}





3 3D coordinates generated are not exact coordinates due to the 
initial position estimate. The x and y positions determined by 
scanning the laser are valid if and only if the initial distance 
between mirror center and target plane is correct. Since this won't 
be the case for most cases, found points should be used to calculate 
the real distance between mirror center and target plane. 


\begin{figure}[!htb]\centering
    \includegraphics*[width = 6cm]{bilder/project/sensor_setup_camera.png}
    \caption{Sensor positions (S1, S2, S3), detected sensor positions (M1, M2, M3) and steering mirror position (O) from camera view.}
    \label{fig:sensor_geometry_camera}

    \includegraphics*[width = 16cm]{bilder/project/sensor_setup_top.png}
    \caption{Sensor positions (S1, S2, S3), detected sensor positions (M1, M2, M3) and steering mirror position (O) from top view.}
    \label{fig:sensor_geometry_top}
\end{figure}




To calculate the distance between mirror center and target plane, 
the geometry of the sensors is used. Sensors are placed on top 
of each other as in Figure 9 and the plate holding the sensors is 
placed perpendicular to ground. This configuration places all the 
sensors at the same distance from mirror center in the z direction. 
Using this constraint, the distance between sensors and mirror 
center in z direction is calculated: 


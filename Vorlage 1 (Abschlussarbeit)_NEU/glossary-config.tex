
%Neuer Eintragstyp
%\newglossary[fog]{formel}{foi}{foo}{Formelzeichen}
\newglossary[flg]{formel}{fyi}{fyg}{Formelverzeichnis}


\renewcommand*{\glspostdescription}{}

\newglossarystyle{lhft}{%
	\renewenvironment{theglossary}%
	{\begin{longtable}{p{.1\linewidth}p{.75\linewidth}}}%
		{\end{longtable}}%
	\renewcommand*{\glossaryheader}{}%
	\renewcommand*{\glsgroupheading}[1]{\initfamily \scalebox{2}{##1} & \\}%
	\renewcommand*{\glsgroupheading}[1]{}%
	\renewcommand*{\glsgroupskip}{}%
	\renewcommand*{\glossaryentryfield}[5]{%		
		\textrm{\textbf{\glstarget{##1}{##2}}} & ##3\glspostdescription##5\\\vspace{0.0cm}}%

%	\renewcommand*{\glossarysubentryfield}[6]{%
	%	& \glstarget{##2}{\strut}##4\glspostdescription\space ##6\\}%
	%\renewcommand*{\glsgroupskip}{ & \\}%
}

%Ein neuer Verzeichnisstil der auch die Einheit mit ausgibt
\newglossarystyle{lhftsymbols}{%  
	% Verzeichnis wird ein 'longtable'  mit 5 Spalten
	\renewenvironment{theglossary}%  
	{ \begin{longtable}{p{8cm}rr}}%  
		{\end{longtable}}%  
	% Kopf der Tabelle
	\renewcommand*{\glossaryheader}{%  
		\bfseries Name & \bfseries Symbol & \bfseries Einheit %
		\\\endhead}%  
	% Kein Abstand zwischen Gruppen
	\renewcommand*{\glsgroupheading}[1]{}%
	%    
	\renewcommand*{\glossaryentryfield}[5]{%  
		%   \glsentryitem{##1}% Entry number if required  
		\glstarget{##1}{##2}% Name
		& ##4 %Symbol
		& \glsentryuseri{##1}%Einheit
		\\% end of row  
	}%
	% The command \glsgroupskip specifies what to do between glossary groups.
	% Glossary styles must redefine this command. (Note that \glsgroupskip
	% only occurs between groups, not at the start or end of the glossary.)
	\renewcommand*{\glsgroupskip}{\relax}
}

\makeglossaries
\makeindex